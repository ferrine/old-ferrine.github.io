\documentclass{beamer}
\usepackage{../../ferres}
\usepackage{../../../math-symbols}
\usepackage{caption}
\usepackage{subcaption}
\logo{
}

\author{Max Kochurov}
\title[Practical Bayes - Bayesian Modeling]{Bayesian Modeling}
\date{2022}
\institute[MSU]{Moscow State University}
\begin{document}
\begin{frame}
	\maketitle
\end{frame}
\begin{frame}{Agenda}
\tableofcontents
\end{frame}
\section{HMC}
\begin{frame}{Sampling from a distribution}
    \begin{columns}[t]
    \begin{column}{0.5\linewidth}
    Conjugate models
    \begin{itemize}
        \item Limited set of applications
        \item Lack of flexibility
        \item They are scalable
    \end{itemize}
    \begin{figure}
        \centering
        \includegraphics[width=0.7\linewidth]{img/normal-distribution}
        \caption{Easy distribution}
    \end{figure}
    \end{column}
    \begin{column}{0.5\linewidth}
    Most models
    \begin{itemize}
        \item No closed form solution
        \item Posterior distributions is complicated
        \item Less scalable
        \item Flexible
    \end{itemize}
    \begin{figure}
        \centering
        \includegraphics[width=0.7\linewidth]{img/zigzag-distribution}
        \caption{Complicated distribution}
    \end{figure}
    \end{column}
    \end{columns}
\end{frame}
\begin{frame}{Hamiltonian Monte Carlo Intuition}
\begin{columns}
\begin{column}{0.5\linewidth}
HMC samples from a complicated distribution
\begin{enumerate}
    \item Ideas from physics
    \item Requires gradient
    \item Requires numerical integration
\end{enumerate}
Tuned HMC converges to the target distribution
\begin{alertblock}{Warning}
I promised a not math heavy course. But this is important for debugging your models.
\end{alertblock}
\end{column}
\begin{column}{0.5\linewidth}
\begin{figure}
    \centering
    \includegraphics[width=\linewidth]{img/funnel_leapfrog}
    \caption{Leapfrog Integration}
\end{figure}
\end{column}
\end{columns}
\end{frame}
\begin{frame}{HMC Distributions}
\begin{columns}
\begin{column}{0.5\linewidth}
\begin{itemize}
    \item $p(\Theta)$ - Target distribution, $\Theta \in \sR^d$ ($\Theta$ aka \textbf{Position})
    \item $p(\Delta\cond\Theta)$ - Momentum distribution, $\Delta \in \sR^d$ \\($\Delta$ aka \textbf{Velocity})
\end{itemize}
Hamiltonian
\begin{equation*}
 H(\Delta, \Theta) = -\log p(\Delta, \Theta)   
\end{equation*}
\begin{block}{Notes}
\begin{itemize}
    \item $p(\Delta\cond\Theta) = \operatorname{Normal}(0, M)$, usually a Normal distibution
    \item $\Delta$ and $\Theta$ have same dimensions
\end{itemize}

\end{block}
\end{column}
\begin{column}{0.5\linewidth}
\begin{figure}
    \centering
    \includegraphics[width=\linewidth]{img/normal-histogram}
    \caption{$p(\Theta) = \operatorname{Normal}(0, 1)$}
\end{figure}
\end{column}
\end{columns}
\end{frame}

\begin{frame}{HMC Differential Equation}
\begin{columns}
\begin{column}{0.5\linewidth}
\begin{align*}
    H(\Delta, \Theta) &= -\log p(\Delta, \Theta)\\
    &=-\log p(\Delta\cond\Theta) &-\log p(\Theta)\\
    &=\underbrace{K(\Delta, \Theta)}_{\text{Kinetic E}} &+  \underbrace{V(\Theta)}_{\text{Potential E}}
\end{align*}
The Physical \textbf{motion} equation
\begin{align*}
    \frac{\partial \Theta}{\partial t} &= \frac{\partial H}{\partial \Delta}\\
    \frac{\partial \Delta}{\partial t} &= -\frac{\partial H}{\partial \Theta}
\end{align*}
\textbf{Motion} preserves total energy $H(\Delta, \Theta)$
\end{column}
\begin{column}{0.5\linewidth}
\begin{figure}
    \centering
    \includegraphics[width=\linewidth]{img/gaussian-trajectory}
    \caption{Momentum, Position trajectories}
\end{figure}
\end{column}
\end{columns}
\end{frame}
\begin{frame}{HMC Divergences}
A divergence is a huge integration error solving the differential equation.
    \begin{alertblock}{When HMC Fails}
    Bad geometry for Hamiltonian
    \end{alertblock}
Bad geometry comes from a lot of things
\begin{columns}[t]
\begin{column}{0.5\linewidth}
\begin{enumerate}
    \item Strong correlations
    \item Narrow funnels in the posterior
    \item Strong likelihood
    \item Non homogeneous posterior 
\end{enumerate}
\end{column}
\begin{column}{0.5\linewidth}
\begin{figure}
    \centering
    \includegraphics[width=0.7\linewidth]{img/funnel_1}
    \caption{Neal's Funnel}
\end{figure}
\end{column}
\end{columns}

\end{frame}
\begin{frame}{HMC Reading Materials}
    \begin{alertblock}{Advanced Reading}
\begin{enumerate}
    \item Interactive \href{https://chi-feng.github.io/mcmc-demo/app.html}{Demo}
    \item A \href{https://colindcarroll.com/2019/04/11/hamiltonian-monte-carlo-from-scratch/}{tutorial} from Colin Carroll
    \item A \href{https://arxiv.org/abs/1701.02434}{paper} from Michael Betancourt
    \item NUTS \href{https://arxiv.org/abs/1111.4246}{paper} from Matthew D. Hoffman, Andrew Gelman
\end{enumerate}
\end{alertblock}
\end{frame}
\section{Cobb-Douglas model}
\begin{frame}{}
    \centering \Huge Example
\end{frame}

\begin{frame}{Toy example - Cobb-Douglas with Simpson Paradox}
\begin{columns}
\begin{column}{0.5\linewidth}
You should all know the Cobb-Douglas function
\begin{equation*}
    Y \approx A \cdot L ^ \beta
\end{equation*}
In our example:
\begin{enumerate}
    \item data has 6 groups (hierarchical)
    \item We know the groups
    \item We know the total factor productivity $A$ is different per group (different equipment)
    \item Labour productivity $\beta$ does not differ much
\end{enumerate}
\end{column}
\begin{column}{0.5\linewidth}
\begin{figure}
    \centering
    \includegraphics[width=\linewidth]{img/example-cobb-douglas-pooled}
    \caption{Example Data (aggregated)}
\end{figure}
\end{column}
\end{columns}
\end{frame}
\begin{frame}{Toy example - Carpet Knitters}
\begin{columns}
\begin{column}{0.5\linewidth}
Let's put more interpretation in the example
\begin{equation*}
    Y_g \approx A_g \cdot L ^ \beta
\end{equation*}
In our example we have a carpet manufacturing plant with 6 workers (groups):
\begin{enumerate}
    \item Workers make different carpets, thus have total factor productivity $A$
    \item Labour productivity $\beta$ is like concentration, the more you work the less productive you are
    \item Workers produce carpets individually
\end{enumerate}
\end{column}
\begin{column}{0.5\linewidth}
\begin{figure}
    \centering
    \includegraphics[width=\linewidth]{img/carpet-photo}
    \caption{Example Y}
\end{figure}
\end{column}
\end{columns}
\end{frame}
\begin{frame}{The paradox}
    \begin{figure}
    \centering
        \begin{subfigure}[l]{0.45\linewidth}
             \centering
             \includegraphics[width=\linewidth]{img/example-cobb-douglas-pooled-reg}
         \end{subfigure}
         \begin{subfigure}[r]{0.5\linewidth}
             \centering
             \includegraphics[width=\linewidth]{img/example-cobb-douglas-reg}
         \end{subfigure}
    \end{figure}
\end{frame}
\section{First model}
\subsection{Setup}
\begin{frame}{One group model}
    Best practices when you start.
    \begin{itemize}
        \item Start with a most simple model
        \visible<2->{
        \begin{itemize}
            \item You have groups, start with one
            \item Make sure priors are well specified, do checks
        \end{itemize}
        }
        \item Make sure simple model converges well
        \visible<3->{
        \begin{itemize}
            \item If one group model fails, all fail
            \item There are simple checks to verify your model samples well
        \end{itemize}
        }
        \item Write a more complex model
        \visible<4->{
        \begin{itemize}
            \item Try several parametrizations
            \item Check how model samples
            \item Compare models (out of scope for now)
        \end{itemize}
        }
    \end{itemize}
\end{frame}
\subsection{Priors}
\begin{frame}{Starting with a simple model}
    To get an idea why we start simple
    \begin{equation*}
    Y_{g=0} \approx A_{g=0} \cdot L ^ \beta
\end{equation*}
\begin{enumerate}
    \item What is prior for $A$?
    \item What is prior for $\beta$?
    \item What is prior predictive for $Y_{g=0}$?
\end{enumerate}
\end{frame}
\begin{frame}{Writing a model}
    \begin{align*}
        Y_{g=0} &\approx A_{g=0} \cdot L ^ \beta\\
        \log Y_{g=0} &\approx \log A_{g=0} + \log L \cdot \beta\\
    \end{align*}
    Introducing distributions
    \begin{align*}
        \log Y_{g=0} &\sim \operatorname{Normal}(\log A_{g=0} + \log L \cdot \beta, \eps)\\
        \eps &\sim \text{???}\\
        \beta &\sim \text{???}\\
        A_{g=0} &\sim \text{???}\\
    \end{align*}
\end{frame}
\begin{frame}{Prior for $\beta$}
    What is a reasonable prior for labour productivity (elasticity) $\beta$? Questions to ask yourself
    \begin{enumerate}
        \item Can it be $<0$? \visible<2->{No}
        \item Can it be large? \visible<3->{No}
        \item Can it be $>1$? \visible<4->{No}
    \end{enumerate}
    \visible<5->{Conclusion: It is bounded by $(0, 1)$
    \begin{block}{The prior is subjective!}
    Who can argue these bounds do not make sense?
    \end{block}}
    \visible<6->{
    \begin{alertblock}{Not yet a prior}
    To get a prior we need a distribution that fits the reasoning
    \end{alertblock}}
\end{frame}
\begin{frame}{Prior for $\beta$}
    What we know: 
    \begin{itemize}
        \item $\beta \in (0, 1)$
        \item Less probable to be close to the boundary
        \item Nothing specific about exact value in the range.
    \end{itemize}
\begin{block}{In the mind}
Enumerate possible distributions that fit the reasoning
\end{block}
\begin{enumerate}
    \item<2-> $\operatorname{Beta}(a, b),\; a>0, b > 0$ with some $a, b$ avoids boundaries
    \item<3-> $\operatorname{LogitNormal}(\mu, \sigma)$ - always avoids boundaries
    \item<4-> $\operatorname{Uniform}(0, 1)$ - a special case of $\operatorname{Beta}(1, 1)$
    \item<5-> $\operatorname{Kumaraswamy}(a, b),\; a>0, b > 0$ you do not need to know that
\end{enumerate}
\end{frame}
\begin{frame}{Visualize your prior}
Before writing a line of code, visualise your prior. What do you like more?
\begin{figure}
    \centering
    \includegraphics[width=.6\linewidth]{img/priors-beta}
    \caption{Visualized Priors}
\end{figure}
You can choose the form with theory in mind
\end{frame}
\begin{frame}{Setting a prior}
    I prefer $\operatorname{LogitNormal}(0, 1)$ in this situation. It has a good functional form.
    \begin{block}{To remember}
    \begin{itemize}
        \item Prior is \textbf{your} modelling choice
        \item The choice has to be motivated
        \item The choice should make sense given practical constraints
        \item You should always be able to defend your choice
        \item \textbf{Prior is what you do not know, the uncertainty}
    \end{itemize}
    \end{block}
\end{frame}
\begin{frame}{The model so far}
    \begin{align*}
        \log Y_{g=0} &\sim \operatorname{Normal}(\log A_{g=0} + \log L \cdot \beta, \eps)\\
        \eps &\sim \text{???}\\
        \beta &\sim \operatorname{LogitNormal}(0, 1)\\
        A_{g=0} &\sim \text{???}\\
    \end{align*}
\end{frame}
\begin{frame}{Prior for $\eps$}
    \begin{block}{Rule of thumb}
    Error term is something small. Usually avoids zero.
    \end{block}
    In our case;
    \begin{itemize}
        \item small is "orders of 0.1-0.5"
    \end{itemize}

    \begin{columns}
    \begin{column}{0.5\linewidth}
    Let it be
    \begin{equation*}
        \eps \sim \operatorname{LogNormal}(-2, 1)
    \end{equation*}
    \end{column}
    \begin{column}{0.5\linewidth}
    \begin{figure}
        \centering
        \includegraphics[width=\linewidth]{img/priors-e}
        \caption{Prior for $\eps$}
    \end{figure}
    \end{column}
    \end{columns}
\end{frame}
\begin{frame}{The model so far}
    \begin{align*}
        \log Y_{g=0} &\sim \operatorname{Normal}(\log A_{g=0} + \log L \cdot \beta, \eps)\\
        \eps &\sim \operatorname{LogNormal}(-2, 1)\\
        \beta &\sim \operatorname{LogitNormal}(0, 1)\\
        A_{g=0} &\sim \text{???}\\
    \end{align*}
\end{frame}
\begin{frame}{Visual Model}
    \begin{figure}
        \centering
        \includegraphics[width=\linewidth]{img/graphviz-model-simple}
        \caption{Graphviz Model}
    \end{figure}
\end{frame}
\begin{frame}{Prior Predictive}
    Prior for $\beta$ is an easy one. We need one for $A_{g=0}$
    \begin{itemize}
        \item<2-> No idea what the prior should be
        \item<3-> We have an idea about $Y$
        \item<4-> $Y$ is positive, thus $A$ is positive
        \item<5-> We have practical range for $Y$, can we infer $A$ at a glance?
        \item<6-> Order of 10s for $Y$ makes sense
        \item<7-> Prior predictive can help
    \end{itemize}
    \visible<8->{
    \begin{block}{Definition}
    Prior predictive is simulated observation model given no data. 
    \end{block}
    \begin{alertblock}{The truth}
    Nobody said setting priors is easy. It is the most work.
    \end{alertblock}
    }
\end{frame}
\begin{frame}{Random prior}
\begin{columns}
\begin{column}{0.5\linewidth}
Why not using e.g.
\begin{equation*}
    A \sim \operatorname{LogNormal}(0, 1)
\end{equation*}
\visible<2->{
\begin{alertblock}{Nonsense}
Workers do not produce 800 carpets per week.
\end{alertblock}}
\end{column}
\begin{column}{0.5\linewidth}
\visible<2->{That's why
\begin{figure}
   \centering
   \includegraphics[width=\linewidth]{img/prior-predictive-cobb-douglas}
   \caption{Prior predictive for Y vs data}
\end{figure}}
\end{column}
\end{columns}
\end{frame}
\section{Prior Predictive Check}
\begin{frame}{Analysing the prior predictive}
    \begin{columns}
\begin{column}{0.6\linewidth}
Getting back to a full model 
\begin{align*}
    \log Y_{g=0} &\sim \operatorname{Normal}(\log A_{g=0} + \log L \cdot \beta, \eps)\\
    \eps &\sim \operatorname{LogNormal}(-2, 1)\\
    \beta &\sim \operatorname{LogitNormal}(0, 1)\\
    A_{g=0} &\sim \operatorname{LogNormal}(0, 1)\\
\end{align*}
\begin{itemize}
    \item We see over dispersion in predictions
    \item Variance may come from $A$ or $\eps$
\end{itemize}
\begin{block}{Actions}
\begin{enumerate}
    \item Try reducing $A$ variance
    \item Try reducing $\eps$ variance
\end{enumerate}
\end{block}
\end{column}
\begin{column}{0.4\linewidth}
What can we read here?
\begin{figure}
   \centering
   \includegraphics[width=\linewidth]{img/prior-predictive-cobb-douglas}
   \caption{Prior predictive for Y vs data}
\end{figure}
\end{column}
\end{columns}
\end{frame}
\begin{frame}{Good prior predictive}
\begin{columns}
    \begin{column}{0.6\linewidth}
    \begin{block}{Seminar}
    You will play with the example at the seminar.
    \end{block}
    A good looking prior predictive was with the definition below
    \begin{align*}
        \log Y_{g=0} &\sim \operatorname{Normal}(\log A_{g=0} + \log L \cdot \beta, \eps)\\
        \eps &\sim \operatorname{LogNormal}(-2, 0.1)\\
        \beta &\sim \operatorname{LogitNormal}(0, 1)\\
        A_{g=0} &\sim \operatorname{LogNormal}(-0.5, 0.1)\\
    \end{align*}
    \end{column}
    \begin{column}{0.4\linewidth}
    \begin{figure}
        \centering
        \includegraphics[width=\linewidth]{img/prior-predictive-cobb-douglas-good}
        \caption{Prior predictive for Y vs data}
    \end{figure}
    \end{column}
\end{columns}
\end{frame}
\begin{frame}{What is a good prior predictive?}
\begin{columns}
\begin{column}{0.6\linewidth}
\begin{itemize}
    \item Prior predictive covers \textbf{reasonable} range for observed data.
    \visible<2->{
    \begin{itemize}
        \item no astronomic speeds
        \item no microscopic distances
        \item no black hole densities
        \item no superpower workers
    \end{itemize}
    }
    \item \textbf{Data is reference}, not your objective.
    \visible<3->{
    \begin{itemize}
        \item do not overfit priors on data.
        \item in 90\% cases you do not need data for prior predictive
        \item in 90\% cases common sense should work just fine
        \item in 10\% cases you can ask experts and adjust the priors
        \item data is your last resort
    \end{itemize}
    }
\end{itemize}
\end{column}
\begin{column}{0.4\linewidth}
\begin{figure}
    \centering
    \includegraphics[width=\linewidth]{img/prior-predictive-cobb-douglas-good}
    \caption{Prior predictive for Y vs data}
\end{figure}
\end{column}
\end{columns}
\end{frame}
\begin{frame}{HMC in action}
    
\end{frame}
\begin{frame}{Sampling}
\begin{columns}
\begin{column}{0.3\linewidth}
After we've checked the priors it is time to sample.
\end{column}
\begin{column}{0.7\linewidth}
\begin{figure}
    \centering
    \includegraphics[width=\linewidth]{img/example-cobb-douglas-trace}
    \caption{Posterior MCMC trace}
\end{figure}
\end{column}
\end{columns}
\end{frame}
\section{Hierarchies}
\subsection{Intro}
\begin{frame}{}
    \centering \Huge Hierarchies
\end{frame}
\begin{frame}{Hierarchies}
Initial data has groups. How to take them in account?
\begin{align*}
        \log Y_{\textbf{g}} &\sim \operatorname{Normal}(\log A_{\textbf{g}} + \log L \cdot \beta, \eps)\\
        \eps &\sim \operatorname{LogNormal}(-2, 1)\\
        \beta &\sim \operatorname{LogitNormal}(0, 1)\\
        A_{\textbf{g}} &\sim \text{???}\\
    \end{align*}
\end{frame}
\begin{frame}{What is Hierarchy?}
    \begin{block}{Hierarchy}
    Once you have similar groups in your data, you have hierarchy.
    \end{block}
    Examples:
    \begin{enumerate}
        \item Countries, Regions
        \item User groups: by age, by profession, etc
        \item Treatment groups
        \item Time dependent effects
        \item Panel Data
    \end{enumerate}
    \begin{block}{Our Example}
    Workers make different carpets and have total factor
productivity $A$
    \end{block}
\end{frame}
\begin{frame}{Treating Hierarchy}
Classical Econometrics view:
    \begin{enumerate}
        \item All the groups are independent. \textbf{Pooled Model}
\begin{equation*}
    y_{k,i} = \alpha + \beta x_{k,i} + \varepsilon_{i,k}
\end{equation*}
        \item Groups have significant differences. \textbf{Fixed Effect Model}
\begin{equation*}
    y_{k,i} = \alpha_{k} + \beta x_{k,i} + \varepsilon_{i,k}
\end{equation*}
        \item Groups have non significant, random differences. \textbf{Random Effects Model}
\begin{equation*}
    y_{k,i} = \alpha + \beta x_{k,i} + u_{k} + \varepsilon_{i,k}
\end{equation*}
    \end{enumerate}
Where
\begin{equation*}
    \E u_{k,i} = 0,\quad \E \varepsilon_{k,i} = 0
\end{equation*}
\end{frame}
\subsection{Bayesian}
\begin{frame}{Bayesian Hierarchy}
In
\begin{equation*}
    y_{k,i} = \alpha + \beta x_{k,i} + u_{k} + \varepsilon_{i,k}
\end{equation*}
Let's rearrange terms
\begin{equation*}
    y_{k,i} = (\alpha + u_{k}) + \beta x_{k,i} + \varepsilon_{i,k}
\end{equation*}
\begin{itemize}
    \item $\alpha$ - population mean
    \item $\alpha_k = \alpha + u_k$ - group mean
\end{itemize}
In a Bayesian analysis we need priors. There is more than one way
\begin{columns}
\begin{column}{0.5\linewidth}
\begin{align*}
    \alpha &\sim \operatorname{Normal}(\bar\mu, \bar\sigma)\\
    u_k &\sim \operatorname{Normal}(0, 1)\\
    \alpha_k & = \alpha + u_k \cdot \sigma
\end{align*}
\end{column}
\begin{column}{0.5\linewidth}
\begin{align*}
    \alpha &\sim \operatorname{Normal}(\bar\mu, \bar\sigma)\\
    \alpha_k &\sim \operatorname{Normal}(\alpha, \sigma)
\end{align*}
\end{column}
\end{columns}
\end{frame}
\subsection{Parametrizations}
\begin{frame}{More on priors}
\begin{columns}[t]
\begin{column}{0.5\linewidth}
Non centered parametrization
\begin{align*}
    \alpha &\sim \operatorname{Normal}(\bar\mu, \bar\sigma)\\
    u_k &\sim \operatorname{Normal}(0, 1)\\
    \alpha_k & = \alpha + u_k \cdot \sigma
\end{align*}
Group specific parameter $u_k$ is disentangled
\end{column}
\begin{column}{0.5\linewidth}
Centered parametrization
\begin{align*}
    \alpha &\sim \operatorname{Normal}(\bar\mu, \bar\sigma)\\
    \alpha_k &\sim \operatorname{Normal}(\alpha, \sigma)
\end{align*}
\end{column}
\end{columns}
\vspace{1em}
$\sigma$ is a measure of group differences
\begin{enumerate}
    \item $\sigma \to 0$: Pooled Model
    \item Small $\sigma$: Random Effects / Partial Pooling
    \item Large $\sigma$: Fixed Effects / Unpooled Model
\end{enumerate}
$\sigma$ interpolates between the models
\end{frame}

\begin{frame}{Degeneracy}
TODO: improve the visualization
    \begin{columns}
    \begin{column}{0.5\linewidth}
    Centered parametrization
\begin{align*}
    \alpha &\sim \operatorname{Normal}(\bar\mu, \bar\sigma)\\
    \alpha_k &\sim \operatorname{Normal}(\alpha, \sigma)
\end{align*}
\begin{alertblock}{Warning}
Centered parametrization creates funnel geometry with few data
\end{alertblock}
    \end{column}
    \begin{column}{0.5\linewidth}
    \begin{figure}
        \centering
        \includegraphics[width=\linewidth]{img/funnel}
        \caption{Divergences appear in the Centered Parametrization}
    \end{figure}
    \end{column}
    \end{columns}
\end{frame}
\begin{frame}{Why Funnel is created?}
    \begin{columns}
    \begin{column}{0.5\linewidth}
    Geometry is important
    \begin{enumerate}
        \item Sampler has adaptive step size
        \item With bad geometry Sampler can't find a good one
    \end{enumerate}
    \end{column}
    \begin{column}{0.5\linewidth}
    \begin{figure}
        \centering
        \includegraphics[width=\linewidth]{img/funnel_1}
        \caption{Funnel Geometry}
    \end{figure}
    \end{column}
    \end{columns}
\begin{block}{Suggested reading}
Read more on reparametrization in \href{https://mc-stan.org/docs/2_18/stan-users-guide/reparameterization-section.html}{Stan's Guide}
\end{block}
\end{frame}
\begin{frame}{Inverted Funnel degeneracy}
    A "nice" parametrization does have issues as well.
    \begin{figure}
        \centering
        \includegraphics[width=0.6\linewidth]{img/inverted_funnel}
        \caption{Inverted Funnel Degeneracy}
    \end{figure}
\begin{alertblock}{Advanced Reading}
Read more from \href{https://betanalpha.github.io/assets/case_studies/hierarchical_modeling.html}{Michael Betancourt}
\end{alertblock}
\end{frame}
\subsection{Priors}
\begin{frame}{Setting a Hierarchical Prior}
    \begin{enumerate}
        \item Start with a Pooled or Single group model
        \visible<2->{
        \begin{itemize}
            \item You get an idea of prior parameter scales
            \item You get a decent model structure
            \item Do not care about predictions
        \end{itemize}
        }
        \item Add Hierarchy
        \visible<3->{
        \begin{itemize}
            \item Decide on which parameters to share
            \item Decide on allowed variability for the rest parameters
            \item Debug divergences, reparametrize if required
        \end{itemize}
        }
    \end{enumerate}
\visible<4>{
\begin{block}{Best Practice}
Do not hard-code the parametrization, toggle it in the code
\end{block}
}
\end{frame}
\begin{frame}{The Cobb-Douglas Case}
    \begin{columns}[t]
    \begin{column}{0.5\linewidth}
    Single group model
     \begin{align*}
        \log Y_0 &\sim \operatorname{Normal}(\log A_0 + \log L \cdot \beta, \eps)\\
        \eps &\sim \operatorname{LogNormal}(-2, 0.1)\\
        \beta &\sim \operatorname{LogitNormal}(0, 1)\\
        A_0 &\sim \operatorname{LogNormal}(-0.5, 0.1)\\
    \end{align*}
    \end{column}
    \begin{column}{0.5\linewidth}
    Hierarchical model
    \begin{align*}
        \log Y_k &\sim \operatorname{Normal}(\log A_k + \log L \cdot \beta, \eps)\\
        \eps &\sim \operatorname{LogNormal}(-2, 0.1)\\
        \beta &\sim \operatorname{LogitNormal}(0, 1)\\
        A_k &\sim \operatorname{LogNormal}(\alert<2>{\log A_{\text{pop}}}, \alert<2>{\sigma_A})\\
        \alert<2>{A_{\text{pop}}}&\alert<2>{\sim \operatorname{LogNormal}(-0.5, 0.1)}\\
        \alert<2>{\sigma_A}&\alert<2>{\sim \operatorname{LogNormal}(-2, 0.1)}
    \end{align*}
    \end{column}
    \end{columns}
\visible<2>{
\begin{block}{Hint}
You can reuse some parameters, just add reasonable variability $\sigma_A$
\end{block}
}
\end{frame}
\section{Discussion}
\begin{frame}{Discussion Time}
Setting priors
    \begin{itemize}
        \item Sometimes you do not have expert knowledge
        \item Sometimes parametrization does not allow you to set a good prior
        \item Sometimes prior predictive depends on many parameters
        \item You are limited in time
        \item Using hyperpriors
    \end{itemize}
\end{frame}
\end{document}
