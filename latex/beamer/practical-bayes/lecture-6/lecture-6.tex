\documentclass{beamer}
\usepackage{../../ferres}
\usepackage{../../../math-symbols}
\usepackage{makecell}
\logo{
\includegraphics[width=2cm]{img/msu-logo.png}
}
%math
\author{Max Kochurov}
\title[Practical Bayes - Customer Lifetime]{Customer Lifetime}
\institute[MSU]{Moscow State University}
\begin{document}
\begin{frame}
    \maketitle
\end{frame}

\begin{frame}{Agenda}
\tableofcontents
\end{frame}
\nologo
\section{Background}
\nocite{CLVMeraldoAntonio1}
\nocite{CLVMeraldoAntonio2}
\subsection{Framework}
\begin{frame}{What is it?}
The model rescribes customer relations, they can be the following
\begin{table}[]
        \centering
        \begin{tabular}{c|c|c}
             &  \alert<2>{\makecell{\textbf{Non-Contractual} \\ Unobserved quit}}&\makecell{\textbf{Contractual}\\Observed quit}\\
             \hline
             \alert<2>{\makecell{\textbf{Continuous}\\non-scheduled}}& \alert<2>{\makecell{e-commerce\\ grossery purchase}}&\makecell{credit cards \\ SIM cards}\\
             \hline
             \makecell{\textbf{Discrete}\\scheduled}& \makecell{event tickets\\weekly magazine}&\makecell{netflix\\gym membership}\\
        \end{tabular}
    \end{table}
\end{frame}

\begin{frame}{Conceptual framework}
Rock solid facts about the model:
\visible<2->{
\begin{itemize}
    \item Users quit relations without notice
    \item Users interact with unspecified and unobserved frequency
\end{itemize}
}
But we need some simplifying assumptions:
\visible<3->{
\begin{itemize}
    \item Once quit, users never return
    \item Unobserved frequency is constant in time
\end{itemize}
}
\visible<4->{
\begin{alertblock}{The challenge}
    Hard to differentiate between customers who have indefinitely churned and those who will return in the future.
\end{alertblock}
}
\visible<5->{
BG-NBD model can help
}
\end{frame}
\subsection{The BG-NBD model}
\begin{frame}{The Intuition of the BG-NBD Model}
    BG-NBD is Beta-Gamma Negative Binomial Distribution
    \begin{itemize}
        \item focuses on predicting the number of transactions
    \end{itemize}
It is a part of the LTV model
\begin{equation*}
    \text{LTV} = \text{\alert<2>{number of transactions}} \times \text{value of transaction}
\end{equation*}
\pause
\begin{itemize}
    \item value of transaction is usually an easy thing to get
    \item we mostly care about number of transactions
\end{itemize}
\end{frame}
\begin{frame}{Example}
\begin{columns}
    \begin{column}{0.5\linewidth}
    \begin{itemize}
        \item You are a bakery owner
        \item You have purchase records of your customers (with id)
        \item You want to plan next year revenue
    \end{itemize}
    \end{column}
    \begin{column}{0.5\linewidth}
    \begin{figure}
        \centering
        \includegraphics[width=\linewidth]{img/bakery.png}
    \end{figure}    
    \end{column}
\end{columns}
\end{frame}
\begin{frame}{Practical Aspects}
\begin{columns}
    \begin{column}{0.5\linewidth}
    In addition to standard assumptions:
    \begin{itemize}
        \item Once quit, users never return
        \item Unobserved frequency is constant in time
    \end{itemize}
    There is couple more:
    \visible<2->{
    \begin{itemize}
        \item<2-> Frequency is user specific
        \item<3-> Quit probability is also user specific
    \end{itemize}}        
    \end{column}
    \begin{column}{0.5\linewidth}
    \begin{figure}
        \centering
        \includegraphics[width=\linewidth]{img/customers.png}
    \end{figure}
    \end{column}
\end{columns}
\visible<4>{
\begin{block}{Interpretation}
So we treat each user separately which is more realistic. \\
The retention follows the same idea.
\end{block}}
\end{frame}
\begin{frame}{How priors help}
    There are number of assumptions we may want to bring in:

    \begin{itemize}
        \item We’re sure that most of our customers make purchases at a rate of 4 purchases a week
        \item Our users are "addicted" to the product and have low quit probability, below 30\% in a year
    \end{itemize}
\begin{block}{Uncertainty}
    Our assumptions may be not very certain and we'll reflect them in the priors
\end{block}
\end{frame}
\begin{frame}{The Poisson Process}
Let's break complicated likelihood in parts.
\begin{itemize}
    \item Users follow Poisson Process to make (discrete) purchases
    \item They make purchases at rate $\lambda=$ 1 and 7 per week.
    \item With this assumption, we can model the time-to-next-purchase $\Delta_t$ as an exponential distribution $\operatorname{Exponential}(\lambda)$
\end{itemize}
\includegraphics[width=\linewidth]{img/prob2purchase.png}
We are certain the Orange customer will make a purchase next day
\end{frame}
\begin{frame}{Time to purchase}
Every gap can be measured with $\operatorname{Exponential}(\lambda)$
\includegraphics[width=\linewidth]{img/tx_visual.png}
\begin{equation*}
    P(t_1, \dots, t_x\cond \lambda) = \prod_{i=1}^{i=x}\lambda e^{-\lambda (t_i-t_{i-1})}=\lambda^xe^{-\lambda t_x}=P(x, t_x\cond \lambda)
\end{equation*}
\begin{block}{Insight}
    You do not need a sequence $t_1,\dots,t_x$, you only need $t_x$ = "Age of the customer at last purchase \textbf{x}" and \textbf{x} = "Number of purchases"
\end{block}
\end{frame}
\begin{frame}{Deactivation Probability}
Every step follows a change to deactivate
\includegraphics[width=\linewidth]{img/deactivation_p.png}
In the previous formula we add $(x-1)$ cases of non deactivation.
\begin{equation*}
    P(x, t_x\cond \lambda, p) = (1-p)^{x-1}\lambda^xe^{-\lambda t_x}
\end{equation*}
\end{frame}
\begin{frame}{Deactivation after $t_x$}
    \begin{enumerate}
        \item The customer did not deactivate after $t_x$, this probability is $(1-p)e^{-\lambda(T-t_x)}$
        \item Customer deactivated with probability $p$ and never returns
    \end{enumerate}
\includegraphics[width=\linewidth]{img/customer_waiting_p.png}
Remainder probability is 
\begin{equation*}
    P(T\cond \lambda, t_x, p) = p + (1-p)e^{-\lambda(T-t_x)}
\end{equation*}
\end{frame}
\begin{frame}{Total Probability}
Step by step we figure out the final formula to use
\begin{align*}
    P(x, t_x, T\cond \lambda, p) &= \underbrace{(p + (1-p)e^{-\lambda(T-t_x)})}_{P(T\cond \lambda, p, t_x)} \times \underbrace{(1-p)^{x-1}\lambda^xe^{-\lambda t_x}}_{P(x, t_x\cond \lambda, p)}\\
    \visible<2->{
    P(x, t_x, T\cond \lambda, p) &= p (1-p)^{x-1}\lambda^xe^{-\lambda t_x}} \\
    \visible<2->{
    &+ (1-p)e^{-\lambda(T-t_x)}(1-p)^{x-1}\lambda^xe^{-\lambda t_x}}\\
    \visible<3->{
    P(x, t_x, T\cond \lambda, p) &= p (1-p)^{x-1}\lambda^xe^{-\lambda t_x} + (1-p)^{x}\lambda^xe^{-\lambda T}
    }
\end{align*}
\visible<3->{
\begin{alertblock}{Important}
This formula is only applicable to customers made at least one purchase, those have $x>0$ and $t_x$. We need a formula for all customers including $x=0$ and absence of $t_x$
\end{alertblock}
}
\end{frame}
\begin{frame}{Total Probability for all customers}
Assumption is that all users after they enroll are active by default.

Given a user without any purchases, we have 
\begin{equation*}
    P(T\cond \lambda, x=0) = e^{-\lambda T}
\end{equation*}
Combining it with the previous formula we have
\begin{equation*}
    P(x, t_x, T\cond \lambda, p) = \begin{cases}
        p (1-p)^{x-1}\lambda^xe^{-\lambda t_x} + (1-p)^{x}\lambda^xe^{-\lambda T}\quad &x>0\\
        e^{-\lambda T}\quad &x=0
    \end{cases}
\end{equation*}
Or, equivalently
\begin{equation*}
    P(x, t_x, T\cond \lambda, p) = \delta_{x>0} \left[p (1-p)^{x-1}\lambda^xe^{-\lambda t_x}\right] + (1-p)^{x}\lambda^xe^{-\lambda T}
\end{equation*}
\end{frame}
\section{Custom Likelihood}
\begin{frame}{Transferring to PyMC}
    There is no such a distribution in PyMC, so you need to implement one. This is what is done with \texttt{\href{https://www.pymc.io/projects/docs/en/stable/api/distributions/generated/pymc.DensityDist.html}{pm.DensityDist}}
\includegraphics[width=0.8\linewidth]{img/densitydist.png}
\end{frame}
\begin{frame}[allowframebreaks]
\frametitle{References}
\bibliographystyle{abbrv}
\bibliography{../../../references.bib}
\end{frame}

\end{document}
